\documentclass[12pt]{article}
\usepackage{graphicx}  
\usepackage{amsmath}   
\usepackage{amssymb}   
\usepackage{listings} 
\usepackage{geometry}
\geometry{left=1.0in,right=1.0in,top=1.0in,bottom=1.0in}
\usepackage{framed}
\usepackage{color}
\usepackage{setspace}
\usepackage[backend=biber, style=apa]{biblatex}
\setlength\bibitemsep{1\baselineskip}
\usepackage[hidelinks,
			colorlinks,
			linkcolor=blue,
			anchorcolor=blue,
			urlcolor=blue,
			citecolor=blue]{hyperref}
\addbibresource{../Reference/reference.bib}
\DeclareCiteCommand{\cite}
  {\usebibmacro{prenote}}
  {\textcolor{blue}{\printnames{labelname},}
   %\setunit{,}
   \textcolor{blue}{\printfield{year}}}
  {\multicitedelim}
  {\usebibmacro{postnote}}

\DeclareCiteCommand{\textcite}
  {\usebibmacro{prenote}}
  {\textcolor{blue}{\printnames{labelname}}
   %\setunit{}
   \textcolor{blue}{(\printfield{year})}}
  {\multicitedelim}
  {\usebibmacro{postnote}}


\title{\textbf{Fundamental Momentum Based \\ Stock Investing}}
\author{
	 Sicheng Wang, Taoying Zhao, Pengcheng Zhou \\
	 (Advisor: Rob Reider, Method Investments \& Advisory Ltd)
	}

\begin{document}
	\maketitle

	Momentum has long been regarded as one of the most important financial anomalies across different asset classes, with significant implications for both academic research and investment strategies.
	Previous studies have been focused on explaining the profitability of momentum strategies (e.g., \cite{chan1996momentum}),
	 exploring the effectiveness of momentum strategy across diverse asset classes and the relationship between momentum and value strategies (see \cite{asness2013value}), 
	 and constructing momentum factor to describe the cross-sectional variation of commodity returns (\cite{bakshi2019understanding}).
	 The construction of traditional momentum is based on asset prices, while recent researches also highlight the importance of the momentum in asset characteristics and economic indicators.
	 For instance, \textcite{boons2019basis} proposes the basis-momentum factor that effectively predicts commodity premiums in both the time
	  series and cross section. Moreover, \textcite{brooks2017half} demonstrates the consistent performance of macro momentum strategies over a half-century.
	
	More importantly, \textcite{NBERw20984} argues that the fundamental momentum (momentum in firm fundamentals) explains the performance of strategies based on 
	price momentum. This finding not only provides strong evidence for identifying the driving factor of the profitability of price momentum strategies, 
	but also opens new avenues for developing investment strategies that integrate price momentum with fundamental momentum.

	Following \textcite{NBERw20984}, we first aim to produce new fundamental momentum strategies based on various firm characteristics (we plan to use WRDS to access the relevant data). 
	Then, we will employ traditional asset pricing techniques, including Fama-MacBeth regression and spanning test,
	 to validate the effectiveness of our strategies. Furthermore, we plan to combine the traditional price momentum strategy with the new fundamental momentum strategy to produce conditional strategy to seek performance enhancement.
	  Finally, we will conduct robustness tests to assess the consistency of our new strategy in different market regimes, particularly during drawdown periods.


	\newpage
	\printbibliography
\end{document}